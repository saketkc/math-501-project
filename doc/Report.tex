%===============================================================================
% $Id: ifacconf.tex 19 2011-10-27 09:32:13Z jpuente $  
% Template for IFAC meeting papers
% Copyright (c) 2007-2008 International Federation of Automatic Control
%===============================================================================
\documentclass{ifacconf}

\usepackage{graphicx}      % include this line if your document contains figures
\usepackage{natbib}        % required for bibliography
\usepackage{amsmath, amsxtra, amsfonts,amscd,amssymb}
\setcounter{tocdepth}{3}
\usepackage{graphicx,wrapfig}
\usepackage{epstopdf}
\usepackage{url}
\usepackage[algo2e,linesnumbered, vlined,ruled]{algorithm2e}
\usepackage{float}
\usepackage{multirow}
\usepackage{mathrsfs}
%===============================================================================
\begin{document}
\begin{frontmatter}



\title{
	Optimal Control Design of a Reparable Multistate system}% \thanksref{footnoteinfo}}    % Title, preferably not more than 10 words.

%\thanks[footnoteinfo]{}

\author[First]{XXX} 
\author[Second]{XXX} 


\address[First]{University of Southern California, 
   LA, CA 90089 USA (e-mail: XXX@ usc.edu).}

\address[Second]{University of Southern California, 
   LA, CA 90089 USA (e-mail: XXX@ usc.edu).}

\begin{abstract}                % Abstract of not more than 250 words.

\end{abstract}

\begin{keyword}

\end{keyword}

\end{frontmatter}

\section{Problem description}
Carefully provide a mathematical description of the problem discussed in this report. 

\section{Methodology}
\label{sec:method}
\begin{equation}
\label{eq1}
\frac{dp_0}{dt} = -\lambda_0 p_0(t) + \int_0^1 \mu_1(x)p_1(x,t)dx + \int_0^1u^{*}(x,t)dx
\end{equation}

\begin{equation}
\label{eq2}
\frac{\partial p_1(x,t)}{\partial t} + \frac{\partial p_1(x,t)}{\partial x} = -\mu_1(x)p_1(x,t) - u^{*}(x,t)
\end{equation}


\textbf{Given Initial Conditions:}

\begin{eqnarray}
p_1(1,t) &=& 0\\
p_0(0) &=& 1
\end{eqnarray}

\textbf{Given Boundary Conditions:}

\begin{eqnarray}
p_1(0,t) &=& \lambda_0 p_0(t)
\end{eqnarray}

\begin{eqnarray}
u^{*}(x,t) &=& (0.3+0.1 sin(x))b(t) 
\end{eqnarray}

And,

\begin{eqnarray}
b(t)+\int_0^1\mu_1(x)f(x,t)dx -0.3p_0^{*}(t) = c(t)\\
f(x,t) = 0.1\cos(\pi t)\sin^2(1-x)\\
p_0^{*}(t) = 0.85+0.05\cos(2\pi t)
\end{eqnarray}



We make couple of substitutions, following the notation that
$z_i^j$ refers to the value of $z$ evaluated at time point $i$ and at position $j$

So 
\begin{eqnarray}
p_0(t_j) = v_j \nonumber \\
p_1(x_i,t_j) = w_j^i \\
\mu_1(x_i) = \mu^i \\
\lambda_ = \lambda_0
\end{eqnarray} 


Using the new notations boundary conditions and initial values:


Initial conditions:
\begin{eqnarray}
w_j^{20} &=& 0\\
v_0 &=& 1
\end{eqnarray}

Boundary Conditions:

\begin{eqnarray}
w_j^0 &=&  \lambda v_j
\end{eqnarray}

Also condensing,


\begin{eqnarray}
u^{*}(x_i,t_j) &=& g^ib_j \\
\int_0^1 u^{*}(x,t_j)dx = Gb_j\\
where\ g^i = (0.3+0.1sin(x_i)) \nonumber
\end{eqnarray}

And,

\begin{eqnarray}
b(t)+\int_0^1\mu_1(x)f(x,t)dx -0.3p_0^{*}(t) = c(t) \nonumber \\
b_j = c_j - F_j \\
where\ F_j = \int_0^1\mu_1(x)f(x,t_j)dx -0.3p_0^{*}(t_j)\\
\end{eqnarray}



Discretizing \ref{eq1}

\begin{eqnarray}
\frac{v_{j+1}-v_j}{\tau} = -\lambda v_j + I_j + I_j^{*} \\
I_j = h[\frac{\mu^0 w_j^0}{2} + \sum_{k=1}^{19} \mu^k w_j^k + \frac{\mu^{20} w_j^{20}}{2} ] \\
&=&  h[\frac{\mu^0 w_j^0}{2} + \sum_{k=1}^{19} \mu^k w_j^k ] \nonumber \\
I_j^{*} = Gb_j
\end{eqnarray}


Discretizing \ref{eq2}
\begin{eqnarray}
\frac{w_{j+1}^i-w_{j}^i}{\tau} + \frac{w_j^{i+1}-w_j^{i-1}}{2h} = -\mu_jw_j^i - g^ib_j \nonumber\\
w_{j+1}^i=w_{j}^i-\frac{\tau}{2h}(w_j^{i+1}-w_j^{i-1})-\tau\mu_j w_j^i - \tau g^i b_j
\end{eqnarray}

Applying LAX scheme $w_j^i = \frac{w_j^{i-1} + w_j^{i+1}}{2}$ ????


From 21:
\begin{eqnarray}
v_{j+1} = (1-\lambda \tau)v_j + \tau I_j + \tau I_j^{*} \\
v_{j+1} = av_j + \tau I_j + \tau I_j^{*}\\
v_{j} = av_{j-1} + \tau I_{j-1} + \tau I_{j-1}^{*}\\
v_{j+1} = a^2v_{j-1} + \tau I_j + a\tau I_{j-1} + \tau I_{j}^{*} + a\tau I_{j-1}^{*} 
\end{eqnarray}


For any $v_{j+1}$ expansion the sum of subscripts of all possible terms should sum to $j$ except the leading term which sums to $j+1$ (trivial) .

Hence:

\begin{equation}
v_{j+1} = a^{n+1}v_0 + \tau I_j + a \tau I_{j-1} + a^2 \tau I_{j-2} + \ldots + \tau I_j^{*} + a\tau I_{j-1}^{*} 
\end{equation}


NOTE: $I_j^{*}$ has $b_j$ term hidden inside. But basically this can be written as a matrix form

\begin{equation}
\begin{pmatrix} a^{n}\tau & a^{n-1}\tau & \cdots & a^2\tau & a\tau& \tau \end{pmatrix} . \begin{pmatrix} I_0^{*} \\ I_1^{*} \\ \cdots \\ I_{j-2}^{*} \\ I_{j-1}^{*} \\ I_{j} ^{*}\end{pmatrix}
\end{equation}

\section{Results}
Illustrate the results using the methodology proposed in Section. \ref{sec:method}

\section{Observation and Conclusions}
State your observation from the results and make conclusions. 

\section*{Reference}
All materials (books, papers, and websites) mentioned in your reports.   
\section*{Appendix}
Attach the Matlab code here.

\end{document}
